\documentclass[lineaire_algebra_oplossingen.tex]{subfiles}
\begin{document}

\chapter{Oefeningen Hoofdstuk 6}

\section{Oefeningen 6.8}


\subsection{Oefening 1}
\subsubsection*{Te Bewijzen}
\[
(\mathbb{R},\mathbb{R}^{2\times 2},+,\langle \cdot,\cdot \rangle ) \text{ is een euclidische ruimte}
\]
Waarbij het inproduct als volgt gedefinieerd is.
\[
\langle A,B \rangle = Tr(A^T\cdot B)
\]

\subsubsection*{Bewijs}
Dit is een heel langdradig bewijs, maar het is hier toch eens volledig uitgeschreven bij wijze van voorbeeld.
\begin{proof}
$(\mathbb{R},\mathbb{R}^{2\times 2},+,Tr)$ is een eindig dimensionale vectorruimte.
We moeten dus enkel de eigenschappen van $\langle A,B \rangle = Tr(A^T,B)$ als inproduct nagaan.\\
Kies willekeurige $V_1,V_2,V, W \in \mathbb{R}^{2\times 2}$ en $\lambda_1,\lambda_2 \in \mathbb{R}^{2\times 2}$.

\begin{itemize}
\item Lineariteit in de eerste component.\\
\[
\langle \lambda_1V_1 + \lambda_2V_2,W \rangle
= Tr((\lambda_1V_1 + \lambda_2V_2)^T\cdot W)
= Tr((\lambda_1V_1^T + \lambda_2V_2^T)\cdot W)
\]
Bovenstaande gelijkheden gelden omwille van de gegeven definitie en de lineariteit van het transponeren van een matrix \footnote{Zie Eigenschap 1.16 p 28.}.
Volgende gelijkheid geldt omdat de spoorafbeelding een lineaire afbeelding is \footnote{Zie Voorbeeld 4.4 p 131 9.}.
\[
= Tr(\lambda_1V_1^T\cdot W + \lambda_2V_2^T\cdot W)
= \lambda_1Tr(V_1^T\cdot W) + \lambda_2Tr(V_2^T\cdot W)
= \lambda_1\langle V_1, W\rangle + \lambda_2\langle V_2, W\rangle
\]
Samengevat staat hierboven het volgende, wat precies betekent dat het gegeven inproduct lineair is in de eerste component.

\item Symmetrie\\
\[
\langle V,W\rangle = Tr(V^T\cdot W) = Tr(V\cdot W^T) = \langle V,W\rangle
\]

\item Positief\\
\[
\langle V,V\rangle = Tr(V^T\cdot V)
= \sum_{i=1}^n(V^T\cdot V)_{ii}
= \sum_{i=1}^n\sum_{j=1}^nV^T_{ij}V_{ji} \ge 0
\]

\item Definiet\\
\[
\langle V,V\rangle
= 0 \Leftrightarrow V=\vec{0}
\]
\begin{itemize}
\item $\Rightarrow$\\
\[
\langle V,V\rangle
= 0 \Rightarrow \sum_{i=1}^n\sum_{j=1}^nV^T_{ij}V_{ji} = 0 \Rightarrow V = \vec{0}
\]

\item $\Leftarrow$\\
\[
v = \vec{0} \Rightarrow \langle V,V\rangle = \langle \vec{0},\vec{0}\rangle
= \sum_{i=1}^n\sum_{j=1}^n\vec{0}_{ij}\vec{0}_{ji} = 0
\]

\end{itemize}

\end{itemize}
\end{proof}
\subsubsection*{Extra}
\[
\Vert A_1 \Vert 
= 1
\]
\[
\Vert A_2 \Vert 
= \sqrt{5}
\]
\[
\Vert A_3 \Vert
= \sqrt{14}
\]
\[
\Vert A_4 \Vert
= 2
\]
\[
\langle A_1,A_2 \rangle
= 2
\]
%TODO Inproducten
\[
\langle A_1,A_3 \rangle
=
\]
\[
\langle A_1,A_4 \rangle
=
\]
\[
\langle A_2,A_3 \rangle
=
\]
\[
\langle A_2,A_4 \rangle
=
\]
\[
\langle A_3,A_4 \rangle
=
\]
%TODO hoeken
\[
\widehat{A_1,A_2}
= \arccos\frac{2}{1\cdot \sqrt{5}}
= \arccos\frac{2\sqrt{5}}{5}
\]
\[
\widehat{A_1,A_3}
=
\]
\[
\widehat{A_1,A_4}
=
\]
\[
\widehat{A_2,A_3}
=
\]
\[
\widehat{A_3,A_4}
=
\]
%TODO afstanden
\[
\Vert A_2-A_1 \Vert
=
\]
\[
\Vert A_3-A_1 \Vert
=
\]
\[
\Vert A_4-A_1 \Vert
=
\]
\[
\Vert A_3-A_2 \Vert
=
\]
\[
\Vert A_4-A_3 \Vert
=
\]

\end{document}