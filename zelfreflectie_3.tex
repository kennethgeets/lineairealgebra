\documentclass[lineaire_algebra_oplossingen.tex]{subfiles}
\begin{document}

\section{Zelfreflectie 3}
\subsection{Oefening 1}
\subsubsection*{a)}
vrij, lineair onafhankelijk
\subsubsection*{b)}
niet vrij, lineair afhankelijk

\subsection{Oefening 2}
De tweede bewering. Deze oefening beschrijft precies wat er mis is met de cursus.

\subsection{Oefening 3}
Ja. Het triviaal voorbeeld is natuurlijk $\{\vec{0}\}$. Als dit niet voldoende is, bestaan er ook nog binaire ringen. Dit zijn ook vectorruimten, als we de bewerkingen goed defini\"eren.

\subsection{Oefening 4}
De deelruimten van de gepunte ruimte zijn ruimtes, vlakken of rechten door de oorsprong, en de deelruimte met enkel de oorsprong. Twee lineair onafhankelijke vectoren spannen een vlak op. \'e\'en vector spant een rechte op. Dire lineair onafhankelijke vectoren spannen opnieuw de ruimte op. 

\subsection{Oefening 5}
Een basis is voortbrengend en vrij (definitie). Noem $\beta = \{v_1,v_2,...,v_n\}$ de basis van $V$.
\begin{proof}
$\beta$ is maximaal vrij want elke vector in $V$ in is een lineaire combinatie van de vectoren in $\beta$. Als \'e\'en van die vectoren aan $\beta$ toegevoegd zou worden, zou $\beta$ dus niet meer vrij zijn. Dit is precies de definitie van maximaal vrij.\\
Stel dat $\beta$ niet minimaal voortbrengend zou zijn zou $\beta$ uitgedunt kunnen worden tot $\beta'$ door er $b$ uit te halen zodat $\beta'$ nog steeds voortbrengend is. Als dit waar is dan zou $b$ een lineaire combinatie zijn van de vectoren uit $\beta'$. Dit is in contradictie met het feit dat $V$ vrij is. $\beta$ is dus minimaal voortbrengend.
\end{proof}

\subsection{Oefening 6}
\subsubsection*{Te bewijzen}
Zij $v_1,v_2,...,v_n \in R^n$ $n$. Zij $A$ de matrix met in de kolommen de vectoren $v_1,v_2,...,v_n \in R^n$.
$v_1,v_2,...,v_n \in R^n$ vormen een basis van $R^n \Leftrightarrow \det(A) \neq 0$.
\subsubsection*{Bewijs}
\begin{proof}
Rechtstreeks bewijs.\\
\emph{$\Rightarrow$}\\
$v_1,v_2,...,v_n$ is een basis van $R^n$ dus $v_1,v_2,...,v_n$ is vrij en voortbrengend. Omdat $v_1,v_2,...,v_n$ vrij is, zal $\det(A)$ niet nul zijn. ($A$ kan niet rijgereduceert worden naar een matrix met een nulrij, want $v_1,v_2,...,v_n$ is vrij.)\\\\
\emph{$\Leftarrow$}\\
Als $\det(A) \neq 0$ dan zal $A$ niet gereduceerd kunnen worden naar een matrix met een nulrij. Dit houdt in dat de kolommen van $A$ lineair onafhankelijk zijn. Omdat het aantal vectoren in de gevormde deelverzameling gelijk is aan $n$, zal deze deelverzameling een basis zijn voor $R^n$.
\end{proof}

\subsection{Oefening 7}
Zij $U$ en $W$ deelruimten van de vectorruimte $V$
\subsubsection*{Te bewijzen}
$dim(U) + dim(W) = dim(U+W)$ $\Leftrightarrow$ $V = U \oplus W$.
\subsubsection*{Bewijs}
\begin{proof}
Rechtstreeks bewijs.\\
\emph{$\Rightarrow$}\\
$dim(U) + dim(W) = dim(U+W)= dim(U+W) + dim(U\cap W)$
\[
dim(U\cap W) = 0
\]
$U \cap W$ is een deelruimte van $V$ en $dim(U\cap W) = 0$ dus $U \cap W = \{\vec{0}\}$.
$U+W = U\oplus W$ (p 99). $U$ en $W$ zijn deelruimten van $V$ dus $U\oplus W$ is een deelruimte van $V$(p 99). Omdat $U$ en $W$ deelruimten zijn van $W$, geldt $dim(U \oplus W) = dim(V)$ (p115). Omdat $dim(U \oplus W) = dim(V)$ en $U \oplus W$ is een deelruimte van $V$ geldt dat $U \oplus V = V$.\\
\emph{$\Leftarrow$}\\
TODO
\end{proof}

\subsection{Oefening 8}
Propositie 3.14 zegt dat elke doorsnede van twee deelruimten een deelruimte is. Neem nu drie deelruimten van $V$. Neem de doorsnede van de eerste twee, en daarvan de doorsnede met de derde. Nu weten we dat voor elke $3$ deelruimten, de doorsnede ervan een deelruimte is. Dit kunnen we itereren voor elke $k \in \mathbb{N}$. Oneindig veel deelruimten geldt dit ook.

\subsection{Oefening 9}
Als $D$ \'e\'en element bevat, is de beschreven verzameling een deelruimte met dimensie $1$. Als $D$ $k+1$ elementen bevat, is het laatste element ofwel een lineaire combinatie van de$k$ andere elementen, ofwel is het er geen. Als het wel een lineaire combinatie van de andere $k$ elementen is, dan is de beschreven verzameling dezelfde als die voor de $k$ andere elementen. Als het geen lineaire combinatie is, dan is het een deelruimte met \'e\'en dimensie meer dan die voor de $k$ andere elementen.

\subsection{Oefening 10}
Ja, maar in onze cursus spreken we niet over oneindige sommen van vectoren, omdat dit concept nog niet gedefinieerd is.

%TODO

\subsection{Oefening 11}
\subsubsection*{Te Bewijzen}
$\{v_1,...,v_n\}$ is vrij $\Leftrightarrow \{co_{\beta}(v_1),...,co_{\beta}(v_n)\}$ is vrij.
\subsubsection*{Bewijs}
\begin{proof}
Rechtstreeks bewijs.\\
Als $\{w_1,...,w_n\}$ is vrij dan geldt de volgende bewering.
\[\sum_{i=1}^n\lambda_iw_i = 0 \Leftrightarrow \forall i \lambda_i = 0\]
\[
\sum_{i=1}^n \lambda_1\sum_{j=1}^m\gamma_j\beta_j = 0 \Leftrightarrow \forall i:\lambda_i=0
\]
Nu proberen we aan te tonen dat de volgende bewering geldt.
\[
\sum_{i=1}^n \lambda_i co_\beta\left(\sum_{j=1}^m\gamma_j\beta_j\right) = 0 \Leftrightarrow \forall i:\lambda_i=0
\]
Neem opnieuw de tweede vergelijking, en neem aan beide kanten de co\"ordinaatafbeelding. Dit kan omdat aan beide kanten elementen uit $V$ staan ($0$ is de nulvector, niet de scalar $0$).
\[
co_{\beta}\left(\sum_{i=1}^n \lambda_1\sum_{j=1}^m\gamma_j\beta_j\right) = co_{\beta}(0) \Leftrightarrow \forall i:\lambda_i=0
\]
De co\"ordinaatafbeelding van de nulvector is opnieuw de nulvector (in een andere vectorruimte welliswaar). Bovendien weten we dat de co\"ordinaatafbeelding een lineaire afbeelding is, dus gaan de som en de scalaire vermenigvuldiging erdoor.
\[
\sum_{i=1}^n \lambda_1 co_{\beta}\left(\sum_{j=1}^m\gamma_j\beta_j\right) = 0 \Leftrightarrow \forall i:\lambda_i=0
\]
\end{proof}

\subsection{Oefening 12}
Zij $A \in \mathbb{R}^{m\times n}$ en $B \in \mathbb{R}^m$.
\subsubsection*{Te Bewijzen}
\[
V = \{ X \in \mathbb{R}^n | AX = B\}
\]
Is een deelruimte van $\mathbb{R}^n$ als en slechts als $B=0$.
\subsubsection*{Bewijs}
\begin{proof}
Rechtstreeks bewijs\\
Als $B=0$ dan is $X$ een lineaire combinatie van de rijen van $A$. De verzameling van alle mogelijke $X$en is dus een deelruimte van $\mathbb{R}^n$.
\end{proof}

\subsection{Oefening 13}
Een basis is vrij en voortbrengend.
\[
D = \{1,X,X^2,...,X^n,...\}
\]
\emph{Vrij}
\begin{proof}
Rechtstreeks bewijs.\\
Intu\"itief is dit makkelijk te zien omdat zuivere veeltermen niet van graad kunnen veranderen door lineaire combinaties te nemen van zuivere veeltermen. Formeler houdt dit het volgende in.
\[
\forall n\in \mathbb{N},\lambda \in \mathbb{R}: \sum_{i=1}^n\lambda_iX^i = a
\] 
Waarbij $a$ \textbf{nooit} een hogere graad kan hebben die groter is dan $n$. Bovendien ka $a$ nooit een $n$ als graad hebben waarvoor er een niet-triviale term in het linker lid staat.
Dat betekent het volgende, waaruit volgt dat de elementen uit $D$ vrij zijn.
\[
\sum_{i=1}^n\lambda_id^i = 0 \Leftrightarrow \forall \lambda_i \in \mathbb{R}: \lambda_i =0
\]
Dit is waar omdat er in $\sum_{i=1}^n\lambda_id^i$ geen enkele (niet-triviale) term staat met dezelfde graad als $0$.
\end{proof}
\emph{Voortbrengend}\\
\begin{proof}
Bewijs uit het ongerijmde.\\
Stel dat $D$ niet voortbrengend is voor $\mathbb{R}[X]$, dan betekent dat het volgende.
\[
\exists v \in \mathbb{R}[X]: \not\exists \lambda_i\in \mathbb{R},d_i \in D: v = \sum_{i=1}^n\lambda_id_i
\]
Informeel gezegd staat hier dat er geen lineaire combinatie van $D$ bestaat die $v$ vormt.
Dit kan enkel waar zijn als $v$ een term heeft met een graad waarvan er geen in $D$ zit. Dit kan niet, omdat er voor elke $n$ een zuivere term van graad $n$ in $D$ zit.
\end{proof}

\subsection{Oefening 14}
Zij $V$ een vectorruimte en $\beta \in V$ een eindige deelverzameling van $V$.
\subsubsection*{Te bewijzen}
As elke vector uit $V$ op een unieke manier geschreven kan worden als een lineaire combinatie van vectoren in $\beta$, dan is $\beta$ een basis van $V$.
Formeler:
\[
\forall v \in V,\ \forall i\in \mathbb{N},\ \exists!\lambda_i\in \mathbb{R},\ \beta_i \in \beta:\ v=\sum_{i = 1}^n\lambda_i\beta_i
\]
\subsubsection*{Bewijs}
\begin{proof}
Een basis is vrij en voortbrengend. We moeten dus bewijzen dat als $\beta$ aan de eigenschap uit het te bewijzen voldoet, dan vrij en voortbrengend is.\\
\emph{vrij}\\
TODO\\
\emph{voortbrengend}\\
TODO
\end{proof}

\subsection{Oefening 15}
In de informatietheorie bestaat er zoiets als een binaire ring.
\[
V = \{0,1\}
\]
Dee optelling en de scalaire vermenigvuldiging zijn als volgt gedefinieerd.
\[
(+) = (\oplus)
\]
\[
(\cdot) = (\wedge)
\]

\end{document}