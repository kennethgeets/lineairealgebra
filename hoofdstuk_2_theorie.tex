\documentclass[lineaire_algebra_oplossingen.tex]{subfiles}
\begin{document}

\chapter{Theorie Hoofdstuk 2}
\section{Bewijzen uit de cursus}

\subsection{Stelling 2.2 p 57}
\label{2.2}
Zij $f : \mathbb{R}^{n\times n} \rightarrow \mathbb{R}:A\mapsto f(A)$ een afbeeldingen die aan definitie 2.1 op pagina 57 voldoet.

\subsubsection*{Te Bewijzen}
\begin{enumerate}
\item $f$ is lineair in de $i$de rij voor elke $i\in \{1,2,...,n\}$
\[
f\left(
\begin{pmatrix}
R_1 \\ \vdots \\ \lambda R_i + \mu R_i' \\ \vdots \\R_n
\end{pmatrix}
\right)
=
\lambda
f\left(
\begin{pmatrix}
R_1 \\ \vdots \\ R_i\\\vdots \\R_n
\end{pmatrix}
\right)
+
\mu 
f
\left(
\begin{pmatrix}
R_1 \\ \vdots \\ R_i' \\\vdots \\R_n
\end{pmatrix}
\right)
\] 
\item Als er in $A$ een nulrij is, of als er twee gelijke rijen zijn, dan geldt $f(A)=0$.
\end{enumerate}

\subsubsection*{Bewijs}
\begin{proof}
\begin{enumerate}
\item We verwisselen rij $1$ met rij $i$ om de volgende uitdrukking te bekomen.
\[
f\left(
\begin{pmatrix}
R_1 \\ \vdots \\ \lambda R_i + \mu R_i' \\ \vdots \\R_n
\end{pmatrix}
\right)
=
-
f\left(
\begin{pmatrix}
\lambda R_i + \mu R_i' \\ \vdots \\ R_1 \\ \vdots \\R_n
\end{pmatrix}
\right)
\] 
We weten dat lineariteit geldt voor de eerste rij\footnote{Zie D-3 in Definitie 2.1 p 57}.
\[
-
f\left(
\begin{pmatrix}
\lambda R_i + \mu R_i' \\ \vdots \\ R_1 \\ \vdots \\R_n
\end{pmatrix}
\right)
=
-
\lambda
f\left(
\begin{pmatrix}
R_i \\ \vdots \\ R_1\\\vdots \\R_n
\end{pmatrix}
\right)
-
\mu 
f
\left(
\begin{pmatrix}
R_i' \\ \vdots \\ R_1 \\\vdots \\R_n
\end{pmatrix}
\right)
\] 
We verwisselen in de matrices in beide termen opnieuw rij $1$ met rij $i$.
\[
\lambda
f\left(
\begin{pmatrix}
R_1 \\ \vdots \\ R_i\\\vdots \\R_n
\end{pmatrix}
\right)
+
\mu 
f
\left(
\begin{pmatrix}
R_1 \\ \vdots \\ R_i' \\\vdots \\R_n
\end{pmatrix}
\right)
\] 
\item
Gevalsonderscheid\\
Geval 1: $A$ heeft twee gelijke rijen.\\
Stel $R_i = R_j$ zijn gelijke rijen. We kunnen nu $R_i$ en $R_j$ omwisselen om $A'$ te bekomen. Nu zal $f(A)$ het tegengestelde zijn van $f(A')$ maar $A$ en $A'$ zijn gelijk omdat de verwisselde rijen gelijk zijn.
\[
f(A) = - f(A) \Leftrightarrow f(A)=0
\]
Geval 2: $A$ heeft een nulrij.\\
Stel dat $R_i$ een nulrij is in $A$. Vervang nu $R_i$ door $R_i'  = \lambda R_i$ (met $\lambda \neq 0$) om $A'$ te bekomen. $A$ is gelijk aan $A'$ want $\forall \lambda \in \mathbb{R}: \lambda*0=0$. Bovendien geldt door lineariteit van de $i$ de rij (zie hierboven) het volgende.
\[
f(A) = \lambda f(A')
\]
Dit kan enkel als $f(A) = 0$.
\end{enumerate}
\end{proof}

\subsection{Stelling 2.3 p 58}
\label{2.3}
Zij $f : \mathbb{R}^{n\times n} \rightarrow \mathbb{R}:A\mapsto f(A)$ een afbeeldingen die aan definitie 2.1 (determinant afbeelding) op pagina 57 voldoet.

\subsubsection*{Te Bewijzen}
\begin{enumerate}
\item Als men op de matrix $A$ een elementaire rijoperatie uitvoert van het type $R_i\rightarrow R_i + \lambda R_j$ met $j\neq i$, dan verandert $f(A)$ niet.
\item Als $E$ een elementaire matrix is, dan is 
\[
\left\lbrace
\begin{array}{c l l}
f(E_1) &= 1 & \text{ als } E_1 \text{ correspondeert met } R_i \rightarrow R_i + \lambda R_j\\
f(E_2) &= -1 & \text{ als } E_2 \text{ correspondeert met } R_i \leftrightarrow R_j\\
f(E_3) &= \lambda & \text{ als } E_2 \text{ correspondeert met } R_i \rightarrow \lambda R_i\\
\end{array}
\right.
\]
\item Als $E$ een elementaire matrix is, geldt steeds dat $f(E\cdot A) = f(E)\cdot f(A)$.
\end{enumerate}

\subsubsection*{Bewijs}
\begin{proof}
\begin{enumerate}
\item We weten dat de afbeelding $f$ lineair is in de $i$-de rij\footnote{Zie Stelling 2.2 p 57 D-4} dus het volgende geldt.
\[
f
\left(
\begin{pmatrix}
R_1\\ \vdots\\R_i+\lambda R_j\\\vdots\\R_j\\\vdots\\R_n
\end{pmatrix}
\right)
=
f
\left(
\begin{pmatrix}
R_1\\ \vdots\\R_i\\\vdots\\R_j\\\vdots\\R_n
\end{pmatrix}
\right)
+
\lambda
f
\left(
\begin{pmatrix}
R_1\\ \vdots\\R_j\\\vdots\\R_j\\\vdots\\R_n
\end{pmatrix}
\right)
=
f(A)
\]
De laatste gelijkheid geldt omdat de matrix in de tweede term van het middelste lid twee gelijke rijen heeft, namelijk $R_i$ en $R_j$\footnote{Zie Stelling 2.2 p 57 D-5}.

\item
Kijk even terug naar pagina 36 om te zien hoe elementaire matrices er uitzien. Onthoud bovendien dat $f(\mathbb{I}_n)=1$ \footnote{Zie definitie 2.1 p 57 D-1}.
\begin{enumerate}
\item $E_1$ bekomen we door op $\mathbb{I}_n$ de elementaire rijoperatie uit te voeren van het type $R_i\rightarrow R_i + \lambda R_j$. In Het eerste deel van dit bewijs dat deze rijoperatie niets aan de waarde van $f$ verandert dus $f(E_1) = f(\mathbb{I}) = 1$.
\item $E_2$ bekomen we door in $\mathbb{I}_n$ twee rijen van plaats te veranderen. Wanneer we twee rijen van plaats veranderen keert het teken van $f$ om dus $f(E_2) = -f(\mathbb{I}_n) = -1$.\footnote{Zie Definitie 2.1 p 57 D-2}\\\\
\item 1$E_3$ bekomen we door in $\mathbb{I}_n$ rij $R_i$ te vervangen door $R_i' = \lambda R_i$. Uit de lineariteit van $f$ volgt dat $f(E_3) = \lambda f(\mathbb{I}_n) = \lambda$\footnote{Zie Stelling 2.2 p 57 D-4}.
\end{enumerate}

\item
We moeten een gevalsonderscheid maken.
\begin{enumerate}
\item $E$ correspondeert met  $R_i \rightarrow R_i + \lambda R_j$.\\
$A' = E\cdot A$ is de matrix waarin $E$ uitgevoerd is op $A$. In het eerste deel van dit bewijs hebben we aangetoond dat de waarde van $f$ gelijk blijft als we $E$ uitvoeren op $A$. Bovendien weten we dat $f(E)=1$ uit deel twee van dit bewijs.
\[
f(E\cdot A) = f(A') = f(A) = f(E)\cdot f(A)
\]
\item $E$ correspondeert met  $R_i \leftrightarrow R_j$.\\
$A' = E\cdot A$ is de matrix waarin twee rijen verwisseld zijn zoals beschreven door $E$. Als we twee rijen verwisselen in een matrix keert het teken van $f$ om\footnote{Zie definitie 2.1 p 57 D-2}.
\[
f(E\cdot A) = f(A') = -f(A) = f(E)\cdot f(A)
\]
De bovenstaande bewering geldt want $f(E) = -1$ zoals bewezen in deel twee van dit bewijs.
\item $E$ correspondeert met $R_i \rightarrow \lambda R_i$.
$A' = E\cdot A$ is de matrix waarin rij $R_i$ vervangen is door $R_i' = \lambda R_i$. Door lineariteit in de $i$-de rij geldt dan de volgende bewering\footnote{Zie Stelling 2.2 p 57 D-4}.
\[
f(A') = \lambda f(A)
\]
\end{enumerate}
\end{enumerate}
\end{proof}


\subsection{Stelling 2.4 p 59}
\label{2.4}
Zij $f : \mathbb{R}^{n\times n} \rightarrow \mathbb{R}:A\mapsto f(A)$ een afbeeldingen die aan definitie 2.1 (determinant afbeelding) op pagina 57 voldoet.

\subsubsection*{Te Bewijzen}
\begin{enumerate}
\item Als $A$ een driehoeksmatrix is, dan is $f(A)$ gelijk aan het product van de diagonaalelementen in $A$.
\[
f(A) = \prod_{i=1}^n (A)_{ii}
\]
\item $A$ is inverteerbaar $\Leftrightarrow f(A) \neq 0$.
\item $\forall A,B \in \mathbb{R}^{n\times n}: f(A\cdot B) = f(A)\cdot f(B)$
\item $f(A^T) = f(A)$.
\end{enumerate}

\subsubsection*{Bewijs}
\begin{enumerate}
\item Als $A$ een driehoeksmatrix is, is $A$ rij-equivalent (uitsluitend via rijoperaties van het type $R_i \rightarrow R_i + \lambda R_j$) met $A'$ zijnde de diagonaalmatrix met precies dezelfde elementen op de diagonaal als $A$.
\[
f(A) = f(A') = \prod_{i=1}^n (A)_{ii}
\]
De bovenstaande bewering volgt uit de lineariteit van de determinant afbeelding en uit de definitie waarin staat dat $f(\mathbb{I}_n)=1$\footnote{Zie Stelling 2.2 p 57 D-4} \footnote{Zie definitie 2.1 p 57 D-2}. $A'$ is namelijk gelijk aan de eenheidsmatrix waar herhaaldelijk de elementaire rijoperatie van type $R_i \rightarrow \lambda R_i$ op is uitgevoerd.
\item Elke inverteerbare matrix valt te bekomen door op de eenheidsmatrix $\mathbb{I}_n$ elementaire rijoperaties uit te voeren zonder ooit een rij te vervangen door een nulrij. Dit houdt in dat elke inverteerbare matrix $A$ geschreven kan worden als volgt.
\[
A = E_1\cdot ... \cdot E_k
\]
Nemen we nu van beide leden de determinant dan krijgen we de volgende uitdrukking.
\[
f(A) = f(E_1\cdot ... \cdot E_k) = f(E_1)\cdot ... \cdot f(E_k) \neq 0
\]
De tweede gelijkheid omdat $\forall A\in \mathbb{R}^{n\times n}: f(E\cdot A) = f(E)\cdot f(A)$\footnote{Zie stelling 2.3 p 58}. De derde gelijkheid geldt omdat geen enkele elementaire matrix een determinant heeft die nul is.\\\\
$A$ is altijd rij-equivalent aan een matrix $U$ in trapvorm\footnote{Zie Propositie 1.8 p 21}. Dit betekent dat er een aantal elementaire matrices $E_1',...,E_k'$ bestaan zodat de volgende bewering geldt.
\[
E_k'\cdot ...\cdot E_1' \cdot A = U
\]
Als $A$ niet inverteerbaar is zal $U$ dat ook niet zijn. Dit zou betekenen dat er op de diagonaal van $U$ een nul staat en volgens deel \'e\'en van dit bewijs is $f(A)$ dan gelijk aan nul.
\item
Als $A$ niet inverteerbaar is, dan is ook $A\cdot B$ niet inverteerbaar. Dan geldt de bewering want volgend deel twee van dit bewijs is $f(A)$ dan gelijk aan nul en $f(A)\cdot f(B) = 0\cdot f(B) = 0$ geldt.\\
Als $A$ wel inverteerbaar is is $A$ het product van elementaire matrices\footnote{Zie stelling 1.36 p 38}.
\[
A = E_1\cdot ...\cdot E_k
\]
Bijgevolg geldt de volgende bewering.
\[
f(A\cdot B) = f(E_1\cdot ...\cdot E_k \cdot B) = f(E_1\cdot ...\cdot E_k) \cdot f(B)) = f(A)\cdot f(B)
\]
De tweede gelijkheid volgt uit het feit dat $\forall A\in \mathbb{R}^{n\times n}: f(E\cdot A) = f(E)\cdot f(A)$ geldt\footnote{Zie stelling 2.3 p 58}.
\item 
Als $A$ inverteerbaar is dan is $A^T$ ook inverteerbaar en omgekeerd\footnote{Zie Opdracht 1.33 p 35}. Als $A$ niet inverteerbaar is geldt $f(A) = f(A^T)=0$. Als $A$ wel inverteerbaar is, is $A$ een product van elementaire matrices\footnote{Zie stelling 1.36 p 38}.
\[
A = E_1\cdot ...\cdot E_k \Leftrightarrow A^T =  E_k^T\cdot ...\cdot E_1^T
\]
Bovenstaande bewering geldt vanwege een eigenschap van de getransponeerde matrix van het product van matrices\footnote{Zie Eigenschap 1.22 p 32 (bovenaan)}. $f(E^T)=f(E)$ want $E$ is ofwel symmetrisch (dan is $E^T$ gelijk aan $E$) ofwel is $f(E)$ gelijk aan  $1$. De volgende bewering geldt dan dus ook.
\[
f(A) = f(E_1\cdot ...\cdot E_k) = f(E_1)\cdot ...\cdot f(E_k) = f(E_1^T)\cdot ...\cdot f(E_k^T) = f(E_k^T\cdot ...\cdot E_1^T) = f(A^T)
\]
\end{enumerate}


\subsection{Gevolg 2.5 p 60}
\label{2.5}
Enkel het tweede puntje is zinvol om te bewijzen. De rest is triviaal of al bewezen.\\
Zij $f$ een determinantafbeelding en $A$ inverteerbaar.

\subsubsection*{Te Bewijzen}
\[
f(A^{-1}) = \frac{1}{f(A)}
\]

\subsubsection*{Bewijs}
\begin{proof}
Rechtstreeks bewijs.\\
\[
A \cdot A^{-1} = \mathbb{I}
\]
\[
f(A \cdot A^{-1}) = f(\mathbb{I})
\]
\[
f(A) \cdot f(A^{-1}) = 1
\]
\[
f(A^{-1}) = \frac{1}{f(A)}
\]
\end{proof}


\subsection{Stelling 2.7 p 61}
\label{2.7}

\subsubsection*{Te Bewijzen}
Elke permutatie $S_n$ is een samenstelling van transposities.

\subsubsection*{Bewijs}
\begin{proof}
Bewijs in woorden.\\
Beschouwen we de permutatie $S_n$, dit is een afbeelding die een originele verzameling $\{x_1,x_2,\dotsc,x_n\}$ afbeeldt op een verzameling $\{y_1,y_2,\dotsc,y_n\}$, waarbij elke $x_i$ uit de eerste verzameling precies overeenkomt met een $y_j$ uit de tweede verzameling met $i,j \in {1,2,\dotsc,n}$. (M.a.w. een permutatie is een bijectie.)\\
Het maximum aantal elementen waarvoor niet geldt $x_k = y_k$ met $k \in {1,2,\dotsc,n}$ is $n$, omdat er maximum $n$ elementen niet op dezelfde plaats staan in de afbeelding als in de originele verzameling.\\
Indien nu geldt dat $x_i = y_i$ voor elke $i,j \in {1,2,\dotsc,n}$, dan is onze verzameling gepermuteerd door nul of meer transposities (indien toegevoegd in de volgende gevallen).\\
Indien geldt dat er maar een paar $x_i, x_j$ bestaat met $i,j \in {1,2,\dotsc,n}$ waarvoor geldt $x_i = y_j$ en $x_j = y_i$ dan kunnen we door transpositie van deze twee elementen onze permutatie bekomen. Onze permutatie wordt dus opgebouwd uit een of meer (indien toegevoegd door volgend geval) permutaties.\\
Indien geldt dat er geldt dat er $k$ elementen niet op hun plaats staan met $2 < k \leq n$ (deze bovengrens bestaat, zoals eerder aangetoond), dan bestaat er een koppel $x_i, x_j$ waarvoor geldt $x_i \neq y_i$ en $x_j \neq y_j$ maar $x_i = y_j$. We kunnen dan $x_i$ en $x_j$ transponeren, waardoor $x_i = x'_j$ wel op zijn plaats staat, $x'_j=y_j$. Er zijn dus nog maximum $k-1$ elementen die niet op hun plaats staan. We kunnen dit proces blijven herhalen door transpositie, tot we in het vorige geval uitkomen indien er nog maar twee elementen niet op hun plaats staan. Dit proces is zeker eindig aangezien $n$ de bovengrens van het beginaantal elementen niet op hun plaats is, en dit aantal bij elke stap met minstens een vermindert.\\
Zo bekomen we dus door nul of meer keer te transponeren vanuit de verzameling $\{x_1,x_2,\dotsc,x_n\}$ de verzameling $\{x'_1,x'_2,\dotsc,x'_n\}$, die gelijk is aan $\{y_1,y_2,\dotsc,y_n\}$. We hebben de permutatie dus gevormd uit een samenstelling van transposities. Dit is mogelijk met elke transpositie $S_n$ met $n \in \mathbb{N}$.
\end{proof}


\subsection{Stelling 2.10 p 62}
\label{2.10}
Zij $\sigma$ een willekeurige permutatie en $\tau$ een transpositie in $\mathbb{S}_n$

\subsubsection*{Te Bewijzen}
\[
sgn(\tau \circ \sigma) = - sgn(\sigma)
\]

\subsubsection*{Bewijs}
\begin{proof}
Bewijs in woorden.\\
Wanneer we op een willekeurige permutatie van $\mathbb{S}_n$ een transpositie uitvoeren, wisselen we twee elementen van plaatst. Nu zijn er twee gevallen. In het eerste gevallen waren die twee elementen nog niet ge\"inverteerd en na de transpositie wel. In het tweede geval waren de twee elementen al ge\"inverteerd. In dat geval zijn de elementen na de transpositie niet meer ge\"inverteerd. In het eerste geval was sgn $sgn(\tau \circ \sigma)$ $1$ voor de transpositie en $-1$ na de transpositie. In het tweede geval was $sgn(\tau \circ \sigma)$ $-1$ voor de transpositie en $1$ na de transpositie. In beide gevallen is het teken dus veranderd.
\[
sgn(\tau \circ \sigma) = - sgn(\sigma)
\]
\end{proof}


\subsection{Stelling 2.12 p 63}
\label{2.12}
Zij $\sigma$ in $\mathbb{S}_n$ een samenstelling van $m \in \mathbb{N}$ transposities.

\subsubsection*{Te Bewijzen}
\[
sgn(\sigma) = (-1)^{m}
\]

\subsubsection*{Bewijs}
\begin{proof}
Bewijs door inductie.\\
\textsf{Basis stap}\\
We bewijzen dat de bewering geldt voor $m=1$. Deze bewering is precies dezelfde als de bewering die bewezen is in stelling 2.10 p 62.\\\\
\textsf{Inductie stap}\\
Stel dat de bewering geldt voor een bepaalde $m \in \mathbb{N}$ dan bewijzen we nu dat ze geldt voor $m=k+1 \in \mathbb{N}$ zodat ze geldt voor alle $m \in \mathbb{N}$.
Nu is $\sigma$ een samenstelling van $k+1$ transposities. $\sigma$ kan ookgezien worden als de samenstelling van een permutatie in $\mathbb{S}_n$ en een transpositie want elke permutatie in $\mathbb{S}_n$ is een samenstelling van transposities.\footnote{Zie Stelling 2.7 p 61}. Noem deze permutatie $\sigma'$. We weten dat $sgn(\sigma') = (-1)^k$ volgens de inductie hypothese. We weten ook dat $sgn(\tau \circ \sigma') = -sign(\sigma')$ waarin $\tau$ de $k+1$de transpositie is \footnote{Zie Stelling 2.10 p 62}. Hieruit volgt dan dat $sgn(\sigma) = (-1)^{k+1}$, en dat is precies wat we moesten bewijze
\end{proof}


\subsection{Stelling 2.18 p 69}
\label{2.18}
Zij $A \in \mathbb{R}^{n\times n}$

\subsubsection*{Te Bewijzen}
\begin{enumerate}
\item
\[
\det(A) = \sum_{j=1}^n(-1)^{i+j}a_{ij}\det(M_{ij})
\]
\item
\[
\det(A) = \sum_{i=1}^n(-1)^{i+j}a_{ij}\det(M_{ij})
\]
\end{enumerate}

\subsubsection*{Bewijs}
\begin{proof}
\[
\det(A) = \sum_{\sigma \in \mathbb{S}_n} a_{1\sigma(1)}a_{2\sigma(2)}...a_{n\sigma(n)}
\]
\begin{enumerate}
\item
\[
= \sum_{i=1}^m a_{ij} (-1)^{i+j}C_{ij}
= \sum_{i=1}^m a_{ij} \det_{M_{ij}}
\]
\item
\[
= \sum_{j=1}^m a_{ij} (-1)^{i+j}C_{ij}
= \sum_{j=1}^m a_{ij} \det_{M_{ij}}
\]
\end{enumerate}
\end{proof}
\subsection{Determinant van Wronski p 70}
\label{wronski}
\subsubsection*{a)}
\textit{Te bewijzen:}
\[
W(x, \sin(x), \cos(x), e^x) = -2e^x(x-1)
\]
\begin{proof}
We stellen de determinant van Wronski op.
\[
W(e^{\lambda_1x}, e^{\lambda_2x}, e^{\lambda_3x})
= \det
\begin{pmatrix}
x &  \sin(x) &  \cos(x)  & e^x \\
1 &  \cos(x) & -\sin(x) & e^x \\
0 & -\sin(x) & -\cos(x) & e^x \\
0 & -\cos(x) &  \sin(x) & e^x
\end{pmatrix}
\]
We ontwikkelen naar de eerste kolom.
\[
= x \cdot \det
\begin{pmatrix}
\cos(x)  & -\sin(x) & e^x \\
-\sin(x) & -\cos(x) & e^x \\
-\cos(x) &  \sin(x) & e^x
\end{pmatrix}
- \det
\begin{pmatrix}
\sin(x)  &  \cos(x)  & e^x \\
-\sin(x) & -\cos(x) & e^x \\
-\cos(x) &  \sin(x) & e^x
\end{pmatrix}
\]
Op de eerste determinant voeren we de ERO $R_3 \longmapsto R_3 + R_1$ uit. Op de tweede determinant voeren we $R_2 \longmapsto R_2 + R_1$ uit.
\[
= x \cdot \det
\begin{pmatrix}
\cos(x)  & -\sin(x) & e^x \\
-\sin(x) & -\cos(x) & e^x \\
0        &  0       & 2e^x
\end{pmatrix}
- \det
\begin{pmatrix}
\sin(x)  &  \cos(x) & e^x \\
0        & 0        & 2e^x \\
-\cos(x) &  \sin(x) & e^x
\end{pmatrix}
\]
We ontwikkelen naar respectievelijk de derde en tweede rij en rekenen verder uit.
\begin{align*}
&=
2xe^x \cdot \det
\begin{pmatrix}
\cos(x)  & -\sin(x) \\
-\sin(x) & -\cos(x)
\end{pmatrix}
+ 2 e^x \cdot \det
\begin{pmatrix}
\sin(x)  &  \cos(x) \\
-\cos(x) &  \sin(x)
\end{pmatrix} \\
&=
2xe^x \cdot
(-\cos^2(x) - \sin^2(x))
+ 2e^x \cdot
(sin^2(x) + cos^2(x))
\end{align*}
Gelukkig bestaat $\sin^2(\alpha) + \cos^2(\alpha) = 1$ zodat we van die sinussen en cosinussen af geraken.
\begin{align*}
&=
- 2xe^x + 2e^x \\
&=
-2e^x(x-1) 
\end{align*}
\end{proof}
\subsubsection*{b)}
\textbf{Er staat een fout in het boek. De opgave hieronder is de juiste opgave.}\\
\textit{Te bewijzen:}
\[
W(e^{\lambda_1x}, e^{\lambda_2x}, e^{\lambda_3x}) = -(\lambda_1-\lambda_2) (\lambda_1-\lambda_3) (\lambda_2-\lambda_3)
e^{(\lambda_1 + \lambda_2 + \lambda_3)x}
\]
\begin{proof}
We stellen de determinant van Wronski op.
\[
W(e^{\lambda_1x}, e^{\lambda_2x}, e^{\lambda_3x})
= \det
\begin{pmatrix}
e^{\lambda_1x}             & e^{\lambda_2x}             & e^{\lambda_3x} \\
\lambda_1 e^{\lambda_1x}   & \lambda_2 e^{\lambda_2x}   & \lambda_3 e^{\lambda_3x} \\
\lambda_1^2 e^{\lambda_1x} & \lambda_2^2 e^{\lambda_2x} & \lambda_3^2 e^{\lambda_3x}
\end{pmatrix}
\]
We voeren de volgende ERO's uit: $R_2 \longmapsto R_2 - \lambda_1R_1$ en $R_3 \longmapsto R_3 - \lambda_1^2R_1$.
\[
= \det
\begin{pmatrix}
e^{\lambda_1x} & e^{\lambda_2x}                           & e^{\lambda_3x} \\
0              & (\lambda_2-\lambda_1) e^{\lambda_2x}     & (\lambda_3-\lambda_1) e^{\lambda_3x} \\
0              & (\lambda_2^2-\lambda_1^2) e^{\lambda_2x} & (\lambda_3^2-\lambda_1^2) e^{\lambda_3x}
\end{pmatrix}
\]
We ontwikkelen naar de eerste kolom.
\[
= e^{\lambda_1x} \cdot \det
\begin{pmatrix}
(\lambda_2-\lambda_1) e^{\lambda_2x}     & (\lambda_3-\lambda_1) e^{\lambda_3x} \\
(\lambda_2^2-\lambda_1^2) e^{\lambda_2x} & (\lambda_3^2-\lambda_1^2) e^{\lambda_3x}
\end{pmatrix}
\]
We voeren de volgende ERO uit: $R_2 \longmapsto R_2 - (\lambda_2+\lambda_1)R_1$
\[
= e^{\lambda_1x} \cdot \det
\begin{pmatrix}
(\lambda_2-\lambda_1) e^{\lambda_2x} & (\lambda_3-\lambda_1) e^{\lambda_3x} \\
0                                    & (\lambda_3^2-\lambda_1^2) e^{\lambda_3x} - (\lambda_2+\lambda_1) (\lambda_3-\lambda_1) e^{\lambda_3x}
\end{pmatrix}
\]
We halen $(\lambda_3-\lambda_1)$ uit de laatste rij en rekenen verder uit.
\begin{align*}
&= (\lambda_3-\lambda_1) e^{\lambda_1x} \cdot \det
\begin{pmatrix}
(\lambda_2-\lambda_1) e^{\lambda_2x} & (\lambda_3-\lambda_1) e^{\lambda_3x} \\
0                                    & (\lambda_3+\lambda_1) e^{\lambda_3x} - (\lambda_2+\lambda_1) e^{\lambda_3x}
\end{pmatrix} \\
&= (\lambda_3-\lambda_1) e^{\lambda_1x} \cdot \det
\begin{pmatrix}
(\lambda_2-\lambda_1) e^{\lambda_2x} & (\lambda_3-\lambda_1) e^{\lambda_3x} \\
0                                    & (\lambda_3-\lambda_2) e^{\lambda_3x}
\end{pmatrix} \\
&= (\lambda_3-\lambda_1) e^{\lambda_1x} \cdot
(\lambda_2-\lambda_1) e^{\lambda_2x} (\lambda_3-\lambda_2) e^{\lambda_3x} \\
&= (\lambda_3-\lambda_1) (\lambda_2-\lambda_1) (\lambda_3-\lambda_2)
e^{(\lambda_1 + \lambda_2 + \lambda_3)x} \\
&= -(\lambda_1-\lambda_2) (\lambda_1-\lambda_3) (\lambda_2-\lambda_3)
e^{(\lambda_1 + \lambda_2 + \lambda_3)x}
\end{align*}
\end{proof}
\subsection{Stelling 2.21 p 71}
\label{2.21}
Zij $A \in \mathbb{R}^{n\times n}$.

\subsubsection*{Te Bewijzen}
\[
A\cdot \text{adj}(A) = \text{adj}(A) \cdot A = \det(A)\cdot\mathbb{I}_n
\]

\subsubsection*{Bewijs}
\begin{proof}
We zullen het matrixproduct $A\cdot \text{adj}(A)$ en $\text{adj}(A) \cdot A$ berekenen.\\\\
\emph{Elementen op de hoofddiagonaal van het product}\\
Het element op de $i$-de rij en de $i$-de kolom van het product ziet er als volgt uit.
\[
(A\cdot \text{adj}(A))_{ii} =
\begin{pmatrix}
a_{i1} & a_{i2}&\cdots&a_{in}\\
\end{pmatrix}
\begin{pmatrix}
C_{i1}\\C_{i2}\\\vdots\\C_{in}
\end{pmatrix}
=
\sum_{k=1}^na_{ik}C_{ik} = \det(A)
\]
Deze uitdrukking geeft de ontwikkeling van $\det(A)$ naar rij $i$ weer.
Analoog geldt voor $\text{adj}(A) \cdot A$ iets gelijkaardig.
\[
(\text{adj}(A) \cdot A)_{ii} =
\begin{pmatrix}
C_{i1}&C_{i2}&\cdots&C_{in}\\
\end{pmatrix}
\begin{pmatrix}
a_{i1}\\a_{i2}\\\cdots\\a_{in}\\
\end{pmatrix}
=
\sum_{k=1}^na_{ik}C_{ik} = \det(A)
\]
\emph{Elementen niet op de hoofddiagonaal van het product}\\
Voor de elementen niet op de hoofddiagonaal rest er ons nu nog te bewijzen dat die nul zijn.\\
We berekenen het element van het product op positie $i\neq j$.
\[
(A\cdot \text{adj}(A))_{ij} =
\begin{pmatrix}
a_{i1} & a_{i2}&\cdots&a_{in}\\
\end{pmatrix}
\begin{pmatrix}
C_{j1}\\C_{j2}\\\vdots\\C_{jn}
\end{pmatrix}
=
\sum_{k=1}^na_{ik}C_{jk}
=0
\]
Deze laatste uitdrukking vormt de ontwikkeling van rij $j$ van de determinant van een matrix $A'$ waarbij $A'$ een matrix is waarbij $R_i = R_j = \begin{pmatrix}a_{i1} & a_{i2}&\cdots&a_{in}\\\end{pmatrix}$. Deze determinant is evident nul omdat er in $A$ twee gelijke rijen voorkomen\footnote{Zie Stelling 2.2 p 57 D-5}.\\
Opnieuw is de uitdrukking voor $\text{adj}(A) \cdot A$ analoog.
\[
(A\cdot \text{adj}(A))_{ij} =
\begin{pmatrix}
C_{j1}&C_{j2}&\cdots&C_{jn}\\
\end{pmatrix}
\begin{pmatrix}
a_{i1}\\a_{i2}\\\vdots\\a_{in}\\
\end{pmatrix}
=
\sum_{k=1}^na_{ik}C_{jk}
=0
\]
\end{proof}


\subsection{Gevolg 2.22 p 72}
\label{2.22}
Zij $A \in \mathbb{R}^{n\times n}$ inverteerbaar.

\subsubsection*{Te Bewijzen}
\[
A^{-1} = \frac{1}{\det(A)}\text{adj}(A)
\]

\subsubsection*{Bewijs}
\begin{proof}
Rechtstreeks bewijs.\\
We weten dat $A\cdot \text{adj}(A) = \text{adj}(A) \cdot A = \det(A)\cdot\mathbb{I}_n$\footnote{Zie Stelling 2.21 p 71}.
\[
\text{adj}(A) \cdot A = \det(A)\cdot\mathbb{I}_n
\]
\[
\text{adj}(A) = \det(A) \cdot A_{-1}
\]
\[
A^{-1} = \frac{1}{\det(A)}\text{adj}(A)
\]
\end{proof}


\subsection{Stelling 2.25 p 74}
\label{2.25}

\subsubsection*{Te Bewijzen}
Door $n+1$ gegeven punten $(x_0,y_0),(x_1,y_1),...,(x_n,y_n)$ met $x_0 < x_1 < ... < x_n$ gaat de grafiek van juist \'e\'en veelterm functie van graad hoogstens $n$.

\subsubsection*{Bewijs}
\begin{proof}
We zullen proberen de gezochte veelterm functie te construeren.
Stel dat de volgende uitdrukking deze veelterm functie voorstelt.
\[
f: \mathbb{R} \rightarrow \mathbb{R} : x \mapsto a_0 + a_1x + a_2x^2+...+a_nx^n
\]
Nu moeten we bewijzen dat de co\"effici\"enten $a_i$ bestaan en uniek zijn zodat het volgende stelsel geldt.
\[
\left\lbrace
\begin{array}{ c }
a_0 + a_1x_0 + a_2x_0^2 + \cdots + a_nx_0^n = y_0\\
a_0 + a_1x_1 + a_2x_1^2 + \cdots + a_nx_1^n = y_1\\
\vdots\\
a_0 + a_1x_n + a_2x_n^2 + \cdots + a_nx_n^n = y_n\\
\end{array}
\right.
\]
Deze co\"effici\"enten bestaan inderdaad en zijn uniek want de volgende determinant is niet nul.
\[
\begin{vmatrix}
1 & x_0 & x_0^2 & \cdots & x_0^n\\
1 & x_1 & x_1^2 & \cdots & x_1^n\\
\vdots & \vdots & \vdots & \ddots & \vdots \\
1 & x_n & x_n^2 & \cdots & x_n^n\\
\end{vmatrix}
\neq 0
\]
\end{proof}


\end{document}