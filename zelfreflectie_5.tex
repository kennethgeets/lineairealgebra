\documentclass[lineaire_algebra_oplossingen.tex]{subfiles}
\begin{document}

\section{Zelfreflectie 5}
\subsection{Oefening 1}
De karacteristieke veelterm van een lineaire afbeelding is onafhankelijk van de gekozen basis. \footnote{Zie Gevolg 5.9 p 182.}
Het spectrum dus ook.

\subsection{Oefening 2}
Fout.
\[
1 \le d(\lambda) \le m(\lambda) \le n
\]

\subsection{Oefening 3}
Verschillende matrixvoorstellingen kunnen bekomen worden door er een matrix van basisverandering mee te vermenigvuldigen.
Die basisveranderingen zijn inverteerbaar. 
Verschillende matrixvoorstellingen zijn dus gelijkvormig.

\subsection{Oefening 4}
Enkel de derde is niet diagonaliseerbaar. De rest heeft een enkelvoudig spectrum.

\subsection{Oefening 5}
Ja, want een elementaire rijoperatie is een inverteerbare afbeelding.

\subsection{Oefening 6}
Neen.

\subsection{Oefening 7}
De kern, de dimensie, de rang. 

\subsection{Oefening 8}
Juist, dat is de definitie van eigenruimte.

\subsection{Oefening 9}
\begin{enumerate}[(a)]
\item Nee. Er zijn wel steeds eigenwaarden en eigenvectoren maar de multipliciteiten van de eigenwaarden zijn niet steeds juist.
\item Juist, duh?
\end{enumerate}

\subsection{Oefening 10}
Dit is een propositie in het boek. (Zie \ref{5.25}.)

\subsection{Oefening 11}
We kunnen niet besluiten dan $L$ diagonaliseerbaar is en bijgevolg ook niets over het spectrum van $L$.


\end{document}